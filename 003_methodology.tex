\section{Methodology}

This study adopts a scoping review methodology following the PRISMA-ScR (Preferred Reporting Items for Systematic Reviews and Meta-Analyses – Scoping Reviews) guidelines. A scoping review approach was selected due to the interdisciplinary nature of the research topic, which spans geospatial data acquisition, remote sensing technologies, and computational analysis techniques applied to mangrove ecosystems. This approach enables comprehensive mapping of existing research, identification of methodological trends, and systematic detection of research gaps without restricting inclusion to narrowly defined outcome measures.

\subsection{Literature Search Strategy}

A structured literature search was conducted across five academic databases: \textit{SpringerLink}, \textit{IEEE Xplore}, \textit{EBSCO}, \textit{Scopus}, and \textit{Nature}. These databases were selected to ensure broad coverage of environmental science, Earth observation, and machine learning literature.

Search queries were designed to capture studies related to mangrove ecosystems, geospatial data acquisition, and computational modelling. Keywords included combinations of terms such as \textit{mangrove}, \textit{remote sensing}, \textit{satellite}, \textit{SAR}, \textit{LiDAR}, \textit{hyperspectral}, \textit{thermal}, \textit{machine learning}, \textit{deep learning}, and \textit{convolutional neural networks}. The search strategy was intentionally broad to maximise recall, consistent with scoping review best practices.

\subsection{Study Selection and Screening}

All records retrieved from the database searches were aggregated and screened in multiple stages. Duplicate records were removed prior to screening. Non-English publications were excluded at this stage to ensure consistency in analysis.

The remaining records underwent title and abstract screening. Because the authors’ institution provides full-text access to all searched databases, no studies were excluded due to retrieval limitations. The screened studies were then divided among members of the research team, and each reviewer independently assessed titles and abstracts for relevance to the research objectives.

Studies were excluded during screening if they:
\begin{itemize}
    \item did not involve geospatial or remotely sensed data.
    \item relied solely on descriptive or basic statistical analyses without computational modelling.
    \item did not employ modelling approaches central to this review, such as machine learning or deep learning methods (e.g., Random Forest, convolutional neural networks).
\end{itemize}

Disagreements between reviewers were resolved through discussion to ensure consistency in study selection.

\subsection{Eligibility Assessment and Inclusion}

Following the screening stage, the remaining studies were assessed at the full-text level for eligibility. Studies were excluded if they lacked a clear spatial component, did not contribute substantively to understanding mangrove ecosystem monitoring or restoration, or fell outside the scope of computational and geospatial analysis considered in this review.

After applying the eligibility criteria, a final set of studies was retained for inclusion in the scoping review. These studies form the basis for the qualitative synthesis and quantitative analysis presented in subsequent sections.

\subsection{Article Selection Process}

The overall study selection process is summarised in the PRISMA flow diagram shown in Figure~\ref{fig:prisma}. The diagram illustrates the identification of records across databases, removal of duplicates, screening of titles and abstracts, assessment of full-text eligibility, and final inclusion of studies in the review.

\begin{figure}[h]
\centering
\includegraphics[width=0.9\linewidth]{Figures/prisma.png}
\caption{PRISMA flow diagram illustrating the identification, screening, eligibility assessment, and inclusion of studies in this scoping review.}
\label{fig:prisma}
\end{figure}
