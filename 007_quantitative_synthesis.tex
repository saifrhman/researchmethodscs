\section{Implications for Mangrove Restoration Research}
The quantitative patterns observed in Tables 1 and 2 provide strong empirical support for the research gaps identified in this review.

First, the dominance of satellite-centric tools combined with classical machine learning techniques indicates that most existing studies prioritise extent mapping and change detection, rather than holistic ecosystem assessment. While effective for monitoring loss and gain, such approaches are insufficient for restoration prioritisation, which requires integrating structural integrity, physiological stress, hydrological suitability, and long-term resilience.

Second, the limited adoption of LiDAR, hyperspectral, thermal, and UAV-based data—together with the scarcity of multimodal computational frameworks—highlights a disconnect between data richness and analytical capability. The sensing technologies required to characterise mangrove ecosystems comprehensively already exist, but are rarely combined within unified analytical pipelines.

Finally, the concentration of computational approaches around a small number of algorithms reinforces concerns regarding methodological inertia. Without systematic exploration of multimodal representation learning, domain adaptation, and generalisable architectures, mangrove research risks remaining fragmented and region-specific.

These findings quantitatively reinforce the need for unified multimodal forest frameworks, in which complementary sensing modalities are fused through flexible computational architectures to produce robust, transferable, and restoration-relevant representations of mangrove ecosystems.