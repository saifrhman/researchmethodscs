\section{Introduction}

\subsection{Mangrove Forest Ecosystems}

Mangrove forests are unique, highly productive coastal ecosystems characterised by salt-tolerant trees and shrubs that thrive in the harsh intertidal zones. These plant species are also known as mangroves, as they are the dominant species in their environments. The areas between land and sea, where conditions fluctuate widely in salinity and water level, create complex niches for the distinct species that can tap into the vast resources that these areas possess; these forests are vital for global ecological balance, providing critical biodiversity services and supporting complex food webs \cite{leal_state_2024}.

A primary function of mangroves is to mitigate global climate change. They possess an exceptional capacity for carbon sequestration, storing up four times more carbon than comparable terrestrial forests \cite{worthington_harnessing_2020}. Also, mangroves serve as natural coastal defence barriers, providing protection against tidal erosion and storm surges, while functioning as habitats for numerous species of fish and birds; thus supporting coastal communities worldwide.

Despite these benefits, mangrove forests are facing rapid decline: deforestation is primarily driven by anthropogenic pressures such as urbanisation, aquaculture expansion, and agricultural development, leading to reported annual loss rates of 1 to 2 per cent of total area \cite{worthington_harnessing_2020}. The development of effective preservation strategies has become increasingly complex due to the diversity of mangrove species and the vast range of global climate conditions, generating demand for globally available datasets and advanced analytical tools. Having access to sound ecological modelling of species distributions allows for more effective restoration efforts, I.E., planting the right mangrove species after logging or climate disasters. Additionally, real-time updates on forest coverage can issue warnings to NGOs and local authorities, which lead to faster, more effective action \cite{leal_state_2024}.  

To address the challenge of global mangrove monitoring, the Global Mangrove Watch (GMW) platform provides open access to remote sensing data and analytical tools. This platform delivers near real-time information on mangrove extent and global change dynamics \href{https://www.globalmangrovewatch.org/}{Global Mangrove Watch}. A key component of GMW is the mangrove loss alert system, which utilises Copernicus Sentinel-2 satellite data processed at a 20-meter resolution. While Sentinel-2 typically provides weekly image acquisitions, the alert process is subject to delays due to cloud cover and the need for multiple observations to confirm genuine ecosystem change. Consequently, an alert is typically transmitted approximately three months after a loss event occurs \cite{leal_state_2024}. 

\subsection{Geological Survery Tools: how is remote sensing performed?}

The main technologies employed to collect geographical data for remote sensing of ecological landscapes are: LiDAR, multispectral imagery and hyperspectral imagery; as seen in Li et al., Giri et al., Lassalle et al., and Ghorbanian et al.\cite{li_mapping_2021}\cite{giri_monitoring_2007}\cite{lassalle_advances_2023}\cite{ghorbanian_weakly_2025}.

LiDAR remote sensing is performed by low altitude aircraft equipped with light lasers; as the aircraft flies at constant speed, the sensor emits light and measures the time it takes to rebound, creating a 3D model of the scanned surface. It may struggle to get accurate plant resolution with multilatered, dense, tall forests\cite{deng_comparison_2023}.

Satellite Aperture Radar (SAR) consists of satellite imagery obtained by using sensors capable of utilising other bands of the electromagnetic spectrum. Depending on the bandwidths used by said sensors the images may be classified as multispectral (about 3 to 10 wide bands within the range of 0.43 um to 12 12.51 um) or hyperspectral (a couple hundred narrow bands in the range of 0.4 to 2.5 um) \cite{gisgeography_multispectral_2014}. The usage of these tools was evaluated by Kwon and collaborators, and yielded promising results\cite{kwon_can_2025}.

Despite having modern technology and open data access, achieving global conservation goals for coastal ecosystems has been difficult because consistent, worldwide data on their past and present health is lacking. Now, access to resources such as the decades-old Landsat satellite archive and newer, higher-resolution satellites enables us to analyse changes over long periods. This high-resolution data helps us accurately map scattered or fragmented ecosystems, such as mangrove forests. This new data is combined with powerful computing systems and cloud platforms (e.g., Google Earth Engine), allowing us to quickly process satellite data for the entire planet. This combination has encouraged many global-scale studies that can help answer critical questions for mangrove conservation. Still, it is important to note that these datasets merge data from different countries and institutions, which is often patchy, has varying resolutions, and may be incomplete. As seen in the \href{https://www.globalmangrovewatch.org/}{Global Mangrove Watch} official documentation, data fusion is needed to some extent, depending on the analysis one wishes to run. It will become evident when the results of the referenced studies are discussed, which poses challenges for running machine learning algorithms and for interpreting results holistically and ecologically.