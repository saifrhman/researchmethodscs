\section{Introduction}

Mangrove forests are unique, highly productive coastal ecosystems characterized by salt-tolerant trees and shrubs that thrive in the harsh intertidal zones—the areas between land and sea where conditions fluctuate widely in salinity and water level. 
These forests are vital for global ecological balance, providing critical biodiversity services and supporting complex food webs.
A primary function of mangroves is their significant role in global climate change mitigation. They possess an exceptional capacity for carbon sequestration, storing up to four times more carbon than comparable terrestrial forests \{worthington_harnessing_2020}. Furthermore, ecologically, mangroves serve as natural coastal defense barriers, providing protection against tidal erosion and storm surges, while also functioning as critical nurseries and habitats for numerous species of fish, crustaceans, and birds, thus supporting coastal communities worldwide.
Despite these indispensable benefits, mangrove forests are facing rapid decline. Deforestation is primarily driven by anthropogenic pressures such as urbanization, aquaculture expansion, and agricultural development, leading to reported loss rates of 1 – 2 percent of the total area per annum \{worthington_harnessing_2020}. The complexity of developing effective preservation strategies is amplified by the sheer diversity of mangrove species and the vast range of global climate conditions they inhabit, necessitating comprehensive, globally available datasets and advanced analytical tools.
To address the challenge of global mangrove monitoring, the Global Mangrove Watch (GMW) platform provides open access to remote-sensing data and analytical instruments. This platform delivers near real-time information on mangrove extent and global change dynamics \href{https://www.globalmangrovewatch.org/}{Global Mangrove Watch}. A key component of GMW is the mangrove loss alert system, which utilizes Copernicus Sentinel-2 satellite data processed at a 20-meter resolution. While Sentinel-2 typically provides weekly image acquisitions, the alert process is subject to delays caused by cloud cover and the requirement for multiple observations to confirm genuine ecosystem change. Consequently, an alert is typically transmitted approximately three months after a loss event occurs \{leal_state_2024}. 
Despite having modern technology and open data access, achieving global conservation goals for coastal ecosystems has been difficult because we lack consistent, worldwide data on their past and present health. Now, access to resources like the decades-old Landsat satellite archive and newer, higher-resolution satellites allows us to analyze changes over long periods. This high-resolution data helps us accurately map scattered or fragmented ecosystems like mangrove forests. This new data is combined with powerful computing systems and cloud platforms (e.g., Google Earth Engine), allowing us to quickly process satellite data for the entire planet. This combination has encouraged many global-scale studies that can help answer critical questions for mangrove conservation. 
