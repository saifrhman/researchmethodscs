\section{Introduction}

Mangrove forests are among the most productive and valuable coastal ecosystems, perfoming ecological functions such as shoreline stabilisation, storm surge attenuation, carbon sequestration, nutrient cycling, and biodiversity support \cite{leal_state_2024}. 

Also, mangroves serve as natural coastal defence barriers, providing protection against tidal erosion and storm surges, while functioning as habitats for numerous species of fish and birds; thus supporting coastal communities worldwide and fishing enterprise globally \cite{leal_state_2024}.

Despite the former, as shown in the The State of the World's Mangroves 2024, these ecosystems continue to experience widespread degradation driven by aquaculture expansion, coastal development, altered hydrological regimes, pollution, and climate induced stressors including sea level rise and increasing temperature extremes\cite{leal_state_2024}.

Over the past two decades, substantial progress has been made in monitoring mangrove ecosystems through advances in Earth observation technologies and computational analysis\cite{lassalle_advances_2023}. Global and regional datasets derived from optical satellite imagery, synthetic aperture radar (SAR), LiDAR, hyperspectral sensors, and unmanned aerial vehicles have enabled increasingly detailed assessments of mangrove extent, structure, and temporal dynamics. 

In parallel, machine learning and deep learning methods have been widely adopted to improve classification accuracy, detect change, and estimate biophysical properties such as canopy height and biomass 
\cite{aparicio_ai-based_2025}, 
\cite{bahaduri_multimodal_2024}, 
\cite{deng_comparison_2023}, 
\cite{ghorbanian_weakly_2025}, 
\cite{giri_monitoring_2007}, 
\cite{hamilton_creation_2016}, 
\cite{jamaluddin_spatialspectraltemporal_2024}, 
\cite{lassalle_advances_2023}, 
\cite{lassalle_deep_2022}, 
\cite{li_mapping_2021}, 
\cite{masancay_spectro-textural_2025}, 
\cite{maung_assessing_2025}, 
\cite{mayamanikandan_mapping_2024}, 
\cite{pimple_enhancing_2023}, 
\cite{sun_synergistic_2025}
\cite{worthington_harnessing_2020}, 
\cite{xie_mangrove_2024}, 
\cite{zhang_fragmangro_2025}. 

Collectively, this body of work has significantly advanced the ability to observe mangrove forests at multiple spatial and temporal scales.

However, these advances have largely evolved in isolation. Existing studies typically focus on specific sensing modalities, geographic regions, or analytical tasks, with limited integration across data sources or modelling approaches. As a result, much of the literature remains oriented toward mapping and monitoring, rather than toward decision relevant assessment of restoration needs. In particular, the growing diversity of datasets and computational tools has not been systematically synthesised to evaluate how effectively current approaches support restoration prioritisation.

To address these challenges, this scoping review systematically synthesises existing research on mangrove ecosystem monitoring and the computational analysis of the resulting data. Specifically, the objectives of this review are to:

\begin{itemize}
\item Examine the completeness and robustness of existing geospatial datasets used for mangrove analysis.\\
\item Identify the sensing technologies employed to acquire mangrove data, including optical, SAR, LiDAR, hyperspectral, and thermal modalities.\\
\item Review the computational and machine learning techniques applied to these datasets, along with their reported performance metrics.\\
\item Outline current limitations associated with data acquisition and computational analysis approaches.\\
\item Identify key research gaps related to regional bias, data availability, and methodological limitations in current mangrove research.\\
\end{itemize}

A scoping review approach is appropriate for this study because the literature on mangrove restoration spans diverse datasets, sensing technologies, and computational methods, with heterogeneous objectives, study designs, and evaluation metrics that preclude narrow systematic synthesis. Thus, this paper aims identify methodological trends and detect research gaps in the current ecological framework used in mangrove ecosystems.

This review does not propose or evaluate a specific computational model: it synthesises existing evidence to identify limitations in current practice and to outline future research directions for mangrove ecosystem analysis.

As will become evident later, strong regional bias limits model generalisation across diverse ecological contexts, and the absence of unified multimodal forest frameworks capable of integrating complementary geospatial data into holistic and decision relevant representations are the main hindrances to streamling the process. This leads to many events such as floods, fires, and human intervention not being addressed in time by local NGOs or local authorities. It also make the implementation of this techniques in new enviroments costly in time and resources.