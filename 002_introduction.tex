\section{Introduction}

Mangrove forests are among the most productive coastal ecosystems, delivering a wide range of ecosystem services that support biodiversity, climate regulation, and human livelihoods. These services include shoreline stabilisation, storm surge attenuation, sediment trapping, fisheries support, and long-term carbon sequestration in biomass and soils (Leal \& Spalding, 2024; Worthington et al., 2020). Mangroves also act as critical buffers against extreme weather events, reducing coastal vulnerability in regions increasingly exposed to climate-driven hazards.

Despite their ecological importance, mangrove forests continue to decline due to anthropogenic pressures such as aquaculture expansion, agriculture, urbanisation, and coastal development, with reported annual loss rates of approximately 1–2\%. These pressures are compounded by climate change, leading to widespread degradation and fragmentation that reduce ecosystem resilience and functionality.

Effective mangrove conservation and restoration require spatially explicit and temporally resolved information on forest extent, condition, and environmental constraints. Advances in satellite remote sensing have enabled near real-time global monitoring, exemplified by the Global Mangrove Watch (GMW) platform, which provides open-access data and mangrove loss alerts derived from Sentinel-2 imagery at 20 m resolution. However, cloud cover and the need for repeated observations mean that loss alerts are typically issued several months after disturbance events.

Although remote sensing has transformed mangrove monitoring from static mapping to data-driven analysis, most studies remain focused on mapping and classification rather than integrated, restoration-oriented decision support. This gap limits the translation of monitoring outputs into timely and effective restoration and management actions.

This scoping review addresses this gap by systematically assessing the current status of geographical data and computational tools used in mangrove conservation and restoration planning. Importantly, this review does not propose or evaluate a specific computational model; rather, it synthesises existing evidence to identify methodological gaps and outline future research directions for multimodal mangrove ecosystem analysis.The objectives are to:

\begin{enumerate}
  \item Review the completeness and robustness of existing mangrove datasets.

  \item Examine technologies used to acquire mangrove data and the ecological attributes they capture.

  \item Analyse computational models and analytical techniques, including their performance metrics and limitations.
  \item Identify and elaborate on key research gaps addressing the limitations and biases present in existing mangrove monitoring studies.
\end{enumerate}