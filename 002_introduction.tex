\section{Introduction}

Mangrove forests are among the most productive and valuable coastal ecosystems, providing essential ecosystem services such as shoreline stabilisation, storm surge attenuation, carbon sequestration, nutrient cycling, and biodiversity support. These services play a critical role in supporting coastal livelihoods and mitigating climate related risks. Despite their importance, mangrove ecosystems continue to experience widespread degradation driven by aquaculture expansion, coastal development, altered hydrological regimes, pollution, and climate induced stressors including sea level rise and increasing temperature extremes.

Over the past two decades, substantial progress has been made in monitoring mangrove ecosystems through advances in Earth observation technologies and computational analysis. Global and regional datasets derived from optical satellite imagery, synthetic aperture radar (SAR), LiDAR, hyperspectral sensors, and unmanned aerial vehicles have enabled increasingly detailed assessments of mangrove extent, structure, and temporal dynamics. In parallel, machine learning and deep learning methods have been widely adopted to improve classification accuracy, detect change, and estimate biophysical properties such as canopy height and biomass. Collectively, this body of work has significantly advanced the ability to observe mangrove forests at multiple spatial and temporal scales.

However, these advances have largely evolved in isolation. Existing studies typically focus on specific sensing modalities, geographic regions, or analytical tasks, with limited integration across data sources or modelling approaches. As a result, much of the literature remains oriented toward mapping and monitoring, rather than toward decision-relevant assessment of restoration needs and long-term ecosystem resilience. In particular, the growing diversity of datasets and computational tools has not been systematically synthesised to evaluate how effectively current approaches support restoration prioritisation.

To address these challenges, this scoping review systematically synthesises existing research on mangrove ecosystem monitoring and restoration prioritisation. Specifically, the objectives of this review are to:
\begin{itemize}
\item Examine the completeness and robustness of existing geospatial datasets used for mangrove analysis.
\item Identify the sensing technologies employed to acquire mangrove data, including optical, SAR, LiDAR, hyperspectral, and thermal modalities.
\item Review the computational and machine learning techniques applied to these datasets, along with their reported performance metrics.
\item Outline current limitations associated with data acquisition and computational analysis approaches.
\item Identify key research gaps related to regional bias, data availability, and methodological limitations in current mangrove research.
\end{itemize}

A scoping review approach is appropriate for this study because the literature on mangrove restoration spans diverse datasets, sensing technologies, and computational methods, with heterogeneous objectives, study designs, and evaluation metrics that preclude narrow systematic synthesis. This approach enables comprehensive mapping of existing evidence, identification of methodological trends, and systematic detection of research gaps across disciplines.

Importantly, this review does not propose or evaluate a specific computational model. Rather, it synthesises existing evidence to identify limitations in current practice and to outline future research directions for mangrove ecosystem analysis. While the paper examines a broad range of methodological and practical issues, such as the availability and reliability of data, limitations in measurement technologies, the ways in which models are assessed, and the extent to which existing approaches support restoration decision making, these challenges ultimately converge around two overarching methodological gaps. The first is strong regional bias that limits model generalisation across diverse ecological contexts. The second is the absence of unified multimodal forest frameworks capable of integrating complementary geospatial data into holistic and decision relevant representations.