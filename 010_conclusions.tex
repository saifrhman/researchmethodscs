\section{Conclusions}

This scoping review synthesised the current state of geographical data, sensing technologies, and computational tools used to determine restoration priorities in mangrove ecosystems. By combining qualitative literature analysis with quantitative synthesis of tools and techniques across major academic databases, the review provides a comprehensive overview of methodological trends, strengths, and limitations in contemporary mangrove research.

The findings reveal that while satellite-based optical and SAR data dominate the literature and classical machine learning approaches remain prevalent, advanced sensing modalities such as LiDAR, hyperspectral, and thermal infrared data are comparatively underutilised. Similarly, despite growing interest in deep learning, most computational pipelines remain CNN-centric and single-modality, limiting their effectiveness in fragmented landscapes and complex ecological settings.

Five critical research gaps were identified. First, strong regional bias in data availability and model development constrains generalisation and scalability. Second, the absence of unified multimodal forest frameworks restricts the integration of complementary structural, functional, and temporal information. Third, CNN-based approaches struggle with small and fragmented mangrove patches and rely heavily on costly manual labelling. Fourth, current sensing strategies inadequately capture submerged and underwater mangrove components. Finally, limited integration of field validation and local ecological knowledge reduces the translational impact of computational outputs.

Addressing these gaps requires a shift toward unified, multimodal, and generalisable computational frameworks that integrate diverse sensing modalities, explicitly account for domain shift, and align model outputs with restoration decision-making needs. By framing mangrove restoration as a multimodal representation learning and generalisation challenge, this review positions future research at the intersection of remote sensing, ecology, and machine learning. Advancing in this direction is essential for translating technological progress into robust, scalable, and socially relevant strategies for mangrove conservation and restoration.
