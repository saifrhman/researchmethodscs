\section*{Findings}

\subsection{Shortcomings of the Computational Analysis Techniques}

This is the section for computational analysis techniques

\subsection{Research Gaps}
Regional Bias

A fundamental gap in current mangrove monitoring and restoration research is the pronounced regional bias in both data availability and computational model development. The majority of mangrove remote sensing studies are geographically concentrated in Asia, particularly China, Southeast Asia, and parts of South Asia, while large mangrove systems in Africa, the Middle East, the Caribbean, and South America remain comparatively underrepresented (Wu et al., 2025; Leal \& Spalding, 2024). This imbalance is driven by unequal access to high-resolution datasets, field validation data, and research infrastructure, rather than by ecological importance.

From a scientific perspective, this bias undermines the generalisability of existing models. Mangrove ecosystems exhibit substantial regional variability in species composition, canopy architecture, sediment type, tidal regime, salinity gradients, and anthropogenic pressure. Models trained in one region implicitly learn region-specific spectral, structural, and phenological signatures. When applied elsewhere, these models often fail to generalise because the underlying relationships between remote sensing signals and ecological properties differ (FragMangro, 2025). For example, mangroves in arid environments (e.g., the Red Sea or Arabian Gulf) display markedly different canopy density, thermal stress signatures, and hydrological dynamics compared to humid tropical mangroves in Southeast Asia, yet most models are not designed to accommodate such variability.

Operationally, regional bias limits the scalability of restoration planning tools. Global initiatives for mangrove restoration and climate mitigation require methods that can be deployed consistently across countries and continents. However, most existing models are trained, validated, and reported within single study areas, with performance metrics that do not reflect behaviour in unseen regions. Only one identified study explicitly explored transfer learning across sites, focusing on biochemical traits derived from hyperspectral data within China (Fu et al., 2025). While this work demonstrates the feasibility of transfer learning, it does not address structural, spatial, or multimodal forest attributes, nor does it test cross-continental generalisation.

Addressing regional bias is therefore essential for both scientific robustness and practical relevance. Future research must incorporate cross-region evaluation, domain adaptation, and transfer learning to ensure that models learn ecologically meaningful representations rather than region-specific artefacts. Without this, AI-driven mangrove monitoring risks reinforcing existing geographic inequities and producing restoration recommendations that are unreliable outside well-studied regions.

Lack of a Unified Multimodal Forest Framework

A second, and arguably more critical, research gap is the absence of a unified multimodal forest framework capable of integrating complementary geospatial data sources into a holistic representation of mangrove ecosystems. Existing studies typically focus on a subset of modalities—such as optical imagery alone, optical combined with SAR, or multispectral data fused with LiDAR—depending on data availability and task specificity (Li et al., 2021; Jamaluddin et al., 2024). While these approaches have demonstrated improvements over single-sensor analyses, they remain fragmented and insufficient for comprehensive restoration prioritisation.

Mangrove ecosystems are inherently multidimensional systems, where forest condition cannot be adequately described by any single modality. Each geospatial data source captures a distinct and essential component of ecosystem structure or function:

Optical multispectral imagery provides information on mangrove extent, canopy cover, phenology, and vegetation indices, enabling detection of deforestation, regrowth, and seasonal dynamics (Bunting et al., 2022).

Synthetic Aperture Radar (SAR) captures vegetation structure, moisture content, and inundation dynamics, offering cloud-independent monitoring and sensitivity to tidal and hydrological processes that strongly influence mangrove viability (Ghorbanian et al., 2025).

LiDAR directly measures three-dimensional canopy architecture, including height, vertical stratification, and aboveground biomass, which are critical indicators of forest maturity, carbon stocks, and restoration success (Li et al., 2021).

Thermal infrared (IR) data reveal surface temperature patterns associated with physiological stress, evapotranspiration, and hydrological alteration, providing early warning signals of degradation that may not be visible in structural or spectral data (Farella et al., 2022).

Hyperspectral or enhanced multispectral data encode biochemical and species-level information, supporting discrimination of mangrove species, functional traits, and nutrient status (Fu et al., 2025).

When analysed in isolation, each modality provides only a partial view of forest condition. Optical imagery may indicate intact canopy cover even when physiological stress is present; SAR may detect structural complexity but not species composition; LiDAR may quantify height without revealing hydrological or thermal constraints. As a result, restoration decisions based on single-modality analyses risk misidentifying suitable or resilient restoration sites.

A unified multimodal framework enables the integration of these complementary signals, producing a holistic forest representation that aligns structural, functional, and temporal dimensions. Such integration is particularly important for restoration prioritisation, which requires understanding not only where mangroves exist, but where they can persist and recover under future environmental conditions.

Recent advances in multimodal deep learning provide promising tools for achieving this integration. Li et al. (2021) demonstrated that fusing multispectral, hyperspectral, and LiDAR data substantially improves the mapping of multilayer mangrove structure, highlighting the value of cross-modal information sharing. Separately, Bahaduri et al. (2024) introduced a Cross-Attention Multimodal Transformer (CA-MMT) that performs fusion at the channel level, allowing each modality to dynamically attend to and align with others based on contextual relevance. Channel-level cross-attention enables stronger interaction between modalities than traditional late-fusion approaches, which typically concatenate features after independent processing and fail to capture inter-modal dependencies.

In the context of mangrove ecosystems, channel-level fusion is particularly advantageous. Data availability and relevance vary spatially and temporally due to cloud cover, tidal state, and sensor coverage. A CA-MMT-style architecture can adaptively weight modalities—for example, relying more heavily on SAR during cloudy periods, on thermal IR during heat stress events, or on LiDAR where structural information is available—while maintaining a coherent latent representation of the forest. This flexibility is essential for real-world deployment across heterogeneous environments.

Importantly, future research should not be constrained to CA-MMT architectures alone. Alternative approaches, including multimodal CNNs, graph-based neural networks capturing spatial connectivity, hybrid ConvLSTM -Transformer pipelines for spatiotemporal modelling, and ensemble learning frameworks, may be equally effective depending on data availability and restoration objectives (Jamaluddin et al., 2024; Wu et al., 2025). The critical gap lies not in the absence of a specific model, but in the lack of end-to-end, restoration-oriented multimodal frameworks designed to integrate diverse data sources into ecologically meaningful outputs.

Addressing this gap would enable a shift from task-specific mapping toward decision-relevant ecosystem assessment, supporting restoration strategies that are resilient, scalable, and grounded in the full complexity of mangrove forest systems.

Convolutional Neural Networks (CNNs) are the most widely used models in the field and form the foundation of most mangrove classification approaches. However, they struggle to detect small and fragmented mangroves resulting in incorrect classifications. This limitation leads to three sub-challenges: the difficulty of distinguishing small patches of mangroves in complex backgrounds; the inefficiency of manual labelling in deep learning model training; and models developed or trained using data from one specific geographic region failing to perform well when applied to a different region with different characteristics. \cite{zhang_fragmangro_2025}

Another limitation is encountered when analysing underwater scanning. Satellite data relies primarily on visible light and some wavelengths which are typically scattered by water, resulting in poor accuracy for underwater mangroves. \cite{pimple_enhancing_2023}

Furthermore, field validation and cross-referencing with local knowledge are necessary as some communities continue to explore methods for partial restoration. This can shape projects to maximise the benefits of identification and conservation of mangroves. \cite{leal_state_2024}
\subsection{Future Work: Toward Unified Multimodal Frameworks for Mangrove Restoration Prioritisation}
Future research should move beyond single-sensor or task-specific pipelines toward unified, multimodal frameworks capable of representing mangrove ecosystems as integrated spatiotemporal, structural, and functional systems. While existing studies have demonstrated the value of combining selected modalities—such as multispectral imagery with LiDAR for structural mapping (Li et al., 2021) or SAR with optical time series for large-area monitoring (Jamaluddin et al., 2024)—these integrations remain fragmented and are not designed to support end-to-end restoration decision-making.

A critical direction for future work is the development of holistic multimodal models that jointly ingest spatiotemporal optical imagery, SAR, LiDAR, thermal infrared (IR), and multispectral or hyperspectral data. Each modality captures a distinct aspect of mangrove ecosystems: optical data provide phenology and extent, SAR captures inundation dynamics and structural backscatter, LiDAR delivers three-dimensional canopy architecture and biomass proxies, thermal IR reveals physiological stress and hydrological anomalies, and hyperspectral data encode biochemical and species-level information (Farella et al., 2022; Li et al., 2021). Integrating these complementary signals within a single modelling framework would enable a more ecologically faithful representation of forest condition and resilience, directly supporting restoration prioritisation.

Recent advances in cross-modal deep learning provide a promising foundation for such integration. In particular, Cross-Attention Multimodal Transformers (CA-MMTs) enable fusion at the channel level, allowing each modality to dynamically attend to and align with others based on contextual relevance (Bahaduri et al., 2024). Unlike late-fusion approaches that concatenate features after independent processing, CA-MMTs explicitly model inter-modal dependencies, improving robustness when individual modalities are noisy or partially missing—a common scenario in mangrove environments affected by cloud cover, tidal inundation, or sensor availability. For example, SAR features may dominate during cloudy periods, while optical or thermal channels may provide greater discriminative power under clear conditions. Channel-level attention allows the model to adaptively weight these inputs rather than treating all modalities equally.

However, future research should not be constrained to CA-MMT architectures alone. Alternative computational strategies—such as multimodal CNNs with shared latent spaces, graph-based neural networks encoding spatial and ecological relationships, hybrid ConvLSTM–Transformer pipelines for temporal modelling, or ensemble frameworks combining physically informed and data-driven models—may be equally or more suitable depending on data availability and restoration objectives (Jamaluddin et al., 2024; Wu et al., 2025). The key requirement is not a specific architecture, but a systematic design philosophy that enables deep interaction across heterogeneous data sources while remaining interpretable and scalable.

Another priority for future work is improving cross-regional generalisation through transfer learning and domain adaptation. Current multimodal studies are almost exclusively trained and evaluated within single regions, limiting their applicability elsewhere. Future frameworks should be explicitly designed to support regional adaptation, for example by pre-training multimodal models on globally distributed mangrove datasets (e.g., Global Mangrove Watch combined with GEDI LiDAR samples) and fine-tuning them with limited local data. Such approaches would help mitigate the strong regional bias identified in the literature and enable more equitable deployment of AI-driven restoration tools (Fu et al., 2025; Wu et al., 2025).

Finally, future multimodal frameworks should move beyond descriptive mapping and incorporate restoration-oriented outputs, such as probabilistic suitability indices, uncertainty estimates, and scenario-based projections under climate and land-use change. Integrating ecological constraints (e.g., elevation thresholds, turbidity limits, and salinity gradients) with multimodal remote sensing outputs would allow models to prioritise restoration sites not only based on current condition but also on long-term viability (Pimple et al., 2023). Embedding such models within cloud-based platforms could further support near-real-time monitoring and decision support for conservation practitioners.

In summary, future work should focus on designing flexible, multimodal, and transferable computational frameworks that integrate diverse geospatial data streams into coherent representations of mangrove ecosystems. Achieving this vision would represent a significant step toward bridging the gap between technological capability and actionable restoration planning, enabling more resilient and effective mangrove conservation strategies at both local and global scales.

% TODO: @saif — consider adding any region-specific observations or local case studies here.
