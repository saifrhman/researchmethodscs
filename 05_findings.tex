\section*{Findings}
This is the section for findings
\subsection{Geological Survery Tools}
This is the section for geological survery tools
\subsection{Computational Analysis Techniques}
This is the section for computational analysis techniques
\subsection{Research Gaps}
<<saif adds more here>>>

Convolutional Neural Networks (CNNs) are the most widely used models in the field and are the foundation of most mangrove classification approaches. 
However, they struggle to detect small and fragmented mangroves resulting in incorrect classifications.  \cite{zhang_fragmangro_2025}
This gap in specific leads to the three sub-challenges, the first one is the difficulty of distinguish small patches of mangroves in complex backgrounds. Secondly, manual labeling in deep learning model training results ineficient. And lastly, models used for analyzing mangroves (e.g., using satellite imagery or other data) that were developed or trained using data from one specific geographic region do not perform well when applied to a different region with different characteristics.

Another limitation encountered was when analysing underwater scanning. Given that satellite data relies mainly on visible light and some wavelenghts, which are typically scattered by water, making accuracy poor for underwater mangroves. \cite{pimple_enhancing_2023}