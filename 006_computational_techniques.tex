\section{Computational Analysis Techniques}
Machine learning and deep learning methods are increasingly used to analyse mangrove-related geospatial datasets for both regression and classification tasks, driven by the growing availability of multi-source remote sensing data and the need to infer biophysical properties at scale \cite{adam_multispectral_2010}\cite{giri_status_2011}, \cite{hamilton_creation_2016}. Regression tasks typically aim to estimate continuous variables such as above-ground biomass, carbon stocks, or canopy height, whereas classification tasks focus on mangrove extent delineation, species discrimination, and change detection \cite{bunting_global_2018}\cite{lucas_potential_2007}. Across these applications, ensemble-based methods, kernel-based learners, and deep neural networks have emerged as dominant computational approaches.
\subsection{Regression-based modelling}
Regression-based machine learning models are widely applied to estimate mangrove biomass and structural attributes from remotely sensed inputs, with Random Forest, Support Vector Machine, and gradient boosting methods among the most commonly used approaches \cite{maung_assessing_2025}, \cite{pradisty_estimating_2025}, \cite{simard_mangrove_2019}. These models are well suited to capturing non-linear relationships between spectral, textural, and structural features \cite{breiman_random_2001} \cite{cortes_support-vector_1995}\cite{friedman_greedy_2001}. Model performance in such studies is typically evaluated using the coefficient of determination (R²) and root mean square error (RMSE)\cite{li_mapping_2021}, \cite{maung_assessing_2025}.

Empirical results reported in the mangrove literature indicate moderate to strong predictive performance for biomass estimation, though with substantial variability across regions and forest types. For example, Maung \cite{maung_assessing_2025} evaluated RF, SVM, and XGBoost models for above-ground biomass estimation in Myanmar mangroves using multispectral and SAR inputs and reported that RF achieved the best performance (R² = 0.48, RMSE = 28.12 Mg ha$^{-1}$) after feature selection, while alternative models exhibited markedly lower explanatory power. In a contrasting subtropical plantation setting, Peng reported higher accuracy using XGBoost applied to Sentinel-2 data, achieving an R² of approximately 0.68 with an RMSE of 6.85 Mg ha$^{-1}$, reflecting the reduced structural complexity and biomass range of managed stands\cite{noauthor_pdf_2025}. In structurally complex old-growth mangroves, Selvaraj reported an R² of 0.78 but a substantially higher RMSE of 38.24 Mg ha$^{-1}$, highlighting the influence of biomass heterogeneity and signal saturation effects \cite{selvaraj_estimating_2023}.

Across studies, reported R² values for mangrove biomass estimation commonly fall within the range of 0.5-0.8, with RMSE values strongly dependent on forest maturity, biomass distribution, and sensor configuration \cite{simard_mangrove_2019} \cite{xie_mangrove_2024}. Multiple studies demonstrate that regression performance improves when complementary structural information is incorporated. In particular, the integration of canopy height metrics derived from airborne or spaceborne LiDAR with optical and SAR features has been shown to mitigate saturation effects and improve predictive accuracy \cite{asner_high-resolution_2012} \cite{simard_mangrove_2019} \cite{xie_mangrove_2024}. These findings consistently support the value of multimodal data fusion for robust estimation of mangrove biophysical properties. 

\subsection{Classification-based modelling}
Classification approaches are extensively used for mangrove extent mapping, species classification, and land cover change detection. Traditional machine learning models such as Random Forest and SVM remain widely adopted due to their robustness to noisy inputs and limited training data requirements \cite{belgiu_random_2016} \cite{giri_status_2011}. More recently, convolutional neural networks (CNNs) and other deep learning architectures have been applied for pixel-level segmentation and fine-scale species mapping, particularly where high-resolution optical or hyperspectral imagery is available \cite{lassalle_advances_2023}, \cite{li_mapping_2021}.

Reported classification performance is typically quantified using overall accuracy, Cohen's Kappa, F1-score, or area under the receiver operating characteristic curve (AUC), depending on task formulation \cite{congalton_assessing_2019}. Representative studies indicate that high classification accuracy is achievable under favourable conditions. For instance, Aparicio demonstrated strong performance of a lightweight CNN architecture (MobileNetV3) for mangrove species identification from smartphone imagery, illustrating the potential of deep learning for in situ and participatory monitoring applications \cite{aparicio_ai-based_2025}. In remote sensing–based studies, Random Forest classifiers frequently achieve overall accuracies exceeding 90\% when trained on high-resolution multispectral or hyperspectral data, while SVM performance is more sensitive to kernel selection and training sample composition \cite{adam_multispectral_2010} \cite{fu_cross-scenario_2025}.

Across the literature, overall classification accuracies for mangrove mapping commonly range between 80\% and 95\%, with higher values often reported in studies covering limited spatial extents or a small number of species classes \cite{lassalle_advances_2023}, \cite{fu_cross-scenario_2025}. However, several studies caution that high overall accuracy can obscure systematic misclassification of rare species or fragmented mangrove patches, emphasising the importance of reporting class-level metrics and confusion matrices \cite{congalton_assessing_2019} \cite{belgiu_random_2016}.

\subsection{Evaluation practices and limitations}
Despite advances in computational modelling, evaluation practices across mangrove studies remain heterogeneous. Most studies rely on random cross-validation or hold-out test sets drawn from the same geographic region used for model training, while cross-region or cross-site validation is rarely conducted \cite{maung_assessing_2025}, \cite{fu_cross-scenario_2025}. As a result, reported performance metrics may overestimate model generalisability, particularly in geographically heterogeneous mangrove systems where spectral, structural, and environmental conditions vary substantially \cite{lucas_potential_2007} \cite{simard_mangrove_2019}.

Differences in target variables, sensor combinations, spatial resolution, and validation protocols further limit direct comparison of reported metrics across studies \cite{congalton_assessing_2019}. Recent work has therefore highlighted the need for standardised benchmarking datasets, spatially explicit validation strategies, and transparent reporting of uncertainty \cite{belgiu_random_2016}\cite{li_mapping_2021}. Ensemble approaches and multimodal learning frameworks have been proposed as promising directions to improve robustness and reduce sensitivity to site-specific biases, although their application in mangrove research remains limited and largely region-specific \cite{bahaduri_multimodal_2024}, \cite{li_mapping_2021}.