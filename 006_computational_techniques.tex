\section{Computational Analysis Techniques}
\subsection{Computational and Analytical Techniques in Mangrove Research}
Table 3 summarises the distribution of computational and analytical techniques applied in mangrove studies.
\begin{table}[ht]
\centering
\begin{tabular}{l r}
\toprule
Computational Technique & Number of Studies \\
\midrule
Spatiotemporal Models     & 634 \\
Random Forest             & 556 \\
Neural Networks (general) & 419 \\
Change Detection          & 350 \\
Time-Series Models        & 305 \\
Support Vector Machines   & 268 \\
Decision Trees            & 192 \\
Maximum Entropy (MaxEnt)  & 79 \\
Gradient Boosting         & 70 \\
XGBoost                   & 70 \\
\bottomrule
\end{tabular}
\caption{Distribution of computational techniques used in mangrove-related studies}
\label{tab:computational-techniques}
\end{table}
The computational landscape revealed by Table-3 is notably methodologically conservative. Classical machine learning methods, particularly Random Forest and Support Vector Machines, dominate the literature. Their prevalence reflects advantages such as interpretability, robustness to noisy inputs, and suitability for moderate-sized datasets, which are common in ecological studies.

Deep learning methods, while increasingly present, remain underrepresented relative to their potential, particularly for tasks requiring spatial context or multimodal integration. Neural networks are often applied in isolation to single-sensor data, typically optical imagery, limiting their ability to model complex ecosystem interactions.

Spatiotemporal and time-series models show growing adoption, reflecting increased access to long-term satellite archives. However, these approaches are typically restricted to single-modality temporal stacks, rather than fully integrated multimodal time-series analysis. Notably, advanced multimodal and attention-based architectures are almost entirely absent, despite their success in broader remote sensing and computer vision domains.
\subsection{Classical Machine Learning Approaches}
Random Forest, Support Vector Machines, and Maximum Entropy models remain widely used due to robustness and interpretability. These methods perform well for mangrove extent mapping and habitat suitability modelling but struggle with continuous structural variables such as canopy height and biomass (Li et al., 2021; Aparicio \& Viodor, 2025).
\subsection{Deep Learning for Mangrove Analysis}
Convolutional Neural Networks have become dominant for high-resolution mangrove mapping, offering improved detection of fragmented and heterogeneous forests (Zhang et al., 2025; FragMangro, 2025). Temporal extensions, including ConvLSTM models, enable modelling of growth, disturbance, and recovery dynamics (Jamaluddin et al., 2024).
\subsection{Multimodal and Attention-Based Models}
Transformer-based architectures have emerged as powerful tools for multimodal remote sensing analysis. Cross-Attention Multimodal Transformers (CA-MMTs) fuse modalities at the channel level, enabling strong alignment and interaction across heterogeneous inputs (Bahaduri et al., 2024). This approach is particularly suited to mangrove ecosystems, where data availability varies spatially and temporally due to cloud cover, tidal state, and sensor coverage.

CA-MMT is highlighted here as a representative example of attention-based multimodal fusion rather than a prescriptive solution, and future work should empirically compare multiple fusion strategies under varying data availability and domain shift conditions.

While classification and regression metrics such as overall accuracy, AUC, and RMSE are widely reported, these metrics do not fully capture ecological or restoration relevance, highlighting the need for task-specific evaluation criteria aligned with resilience, suitability, and long-term viability.

[Figure 3 placeholder: Computational approaches and multimodal fusion strategies]