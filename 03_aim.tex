\section{Aim}
The aim of this paper is to assess the status of the technologies and analysis techniques used in determining the conservation status and restoration needs of mangrove forests; 
as well as identifying research gaps.
\subsection{Objectives}
The five main objectives of this paper are:
\begin{enumerate}
\item To review and assess the current completeness and robustness of various available datasets. Several factors can affect data access and analysis, such as cloud cover and shadows when comparing satellite data. These limitations can hinder the consistency of images required for continuous change monitoring.\cite{pimple_enhancing_2023}
\item To identify the most common technologies used for mangrove identification and conservation, which implies understanding the benefits and limitations of each.
\item To identify the current computational models used to analyse these datasets and assess their performance metrics.
\item To outline the current challenges of measurement tools and computational analysis methods, considering recent datasets and their analysis results. \cite{li_mapping_2021}
\item To elaborate on the existing research gaps in order to address succesfully identified shortcomings, which will be useful and relevant for future work in the sbject.
\end{enumerate}



