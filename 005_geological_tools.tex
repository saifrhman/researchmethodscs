\section{Geological Survey Tools and Data Acquisition Technologies}

Mangrove ecosystems are monitored using a diverse suite of geospatial sensing technologies, each capturing distinct and complementary attributes of forest condition (physiological stress, drought, etc,) or structure (plant height and density, topological features, etc.)\cite{aparicio_ai-based_2025}, \cite{leal_state_2024}. No single sensing modality among the retrieved papers provides a complete description of mangrove ecosystems, making multimodal integration a notable hole in the field, and a milestone for conservation science.

Thus, a holistic representation would refer to a joint latent encoding that simultaneously captures forest extent, vertical structure, physiological condition, hydrological dynamics, and temporal change, enabling restoration prioritisation based on both current state and predicted developments.

Figure 2 provides an overview of the major sensing modalities used in mangrove studies and the ecological attributes they capture. The matrix highlights that individual sensing modalities provide partial and complementary views of mangroves.

\begin{figure}[h]
\centering
\includegraphics[width=0.9\linewidth]{Figures/SensingModalities.png}
\caption{Conceptual modality attribute matrix illustrating the relative strength of ecological information captured by different sensing modalities in mangrove ecosystem studies\cite{giri_status_2011}\cite{zhu_integrating_2019}\cite{pham_mangrove_2019}\cite{lucas_potential_2007}\cite{bunting_global_2018}\cite{simard_mapping_2006}\cite{potapov_mapping_2021}\cite{farella_thermal_2022}\cite{fu_improving_2025}.}
\label{fig:sensing_modalities}
\end{figure}


\subsection{Geospatial and Remote Sensing Tools Used in Mangrove Research}
Table 2 summarises the aggregated frequency of major geospatial and sensing tools used in mangrove studies across the surveyed databases.

\begin{table}[ht]
\centering
\begin{tabular}{l r}
\hline
Tool / Data Source & Number of Studies \\
\hline
High-Resolution Aerial Imagery  & 2431 \\
ALOS-PALSAR (SAR)               & 2205 \\
Thermal SAR (T-SAR)             & 1799 \\
Thermal Infrared Sensors (TIRS) & 1120 \\
Hyperspectral Imaging (HSI)     & 812 \\
SAR (general)                   & 449 \\
Satellite Platforms (general)   & 266 \\
CubeSats                        & 174 \\
LiDAR                           & 123 \\
Landsat                         & 28 \\
\hline
\end{tabular}
\caption{Summary of tools and data sources used in mangrove studies}
\label{tab:tools}
\end{table}

Several clear trends emerge from Table 2. First, satellite-based and airborne optical and radar systems dominate mangrove research. High-resolution aerial imagery and SAR-based platforms (particularly ALOS-PALSAR) appear most frequently. This is likely due to the need to assess large geographical spaces and relying on inundation dynamics to classify mangrove ecosystems.

Noticeably, advanced sensing modalities such as LiDAR and hyperspectral imaging remain comparatively underrepresented, despite their demonstrated ability to capture three-dimensional structure, species composition, and biochemical traits \cite{li_mapping_2021}, \cite{fu_cross-scenario_2025}. This suggests that while high-fidelity structural and functional data are recognised as valuable, their integration into mainstream mangrove monitoring workflows remains limited, likely due to cost, availability, and analytical complexity.

Third, the relatively low representation of Landsat compared to SAR-based and aerial platforms reflects a gradual shift away from coarse optical imagery toward sensors better suited to cloud-prone dynamic coastal environments. Overall, these patterns indicate a strong dependence on a narrow subset of sensing technologies, which further states the need for multimodal integration to capture the full complexity of mangrove ecosystems.

\subsection{Optical Multispectral Remote Sensing}
Optical multispectral imagery from platforms such as Landsat and Sentinel-2 remains the most widely used data source for mangrove monitoring. These sensors support vegetation indices, phenological analysis, and land-cover classification \cite{bunting_global_2022}. Sentinel-2's red-edge bands enhance discrimination between mangrove vegetation and adjacent land covers but remain susceptible to cloud cover and atmospheric interference \cite{zhang_fragmangro_2025}.

\subsection{Synthetic Aperture Radar (SAR)}

Synthetic Aperture Radar (SAR) is an active remote sensing technology operating in the microwave domain, enabling cloud-independent observation and consistent data acquisition in tropical regions. SAR backscatter is sensitive to vegetation structure, moisture content, and inundation dynamics, making it well suited for mangrove delineation and monitoring in persistently cloudy coastal environments. Sensors such as Sentinel-1 and ALOS PALSAR have demonstrated strong performance in detecting narrow and fragmented mangrove stands and in capturing tidal and hydrological variability, which is particularly relevant for restoration suitability assessment \cite{bunting_global_2022, kwon_2020}.

\subsection{LiDAR}
LiDAR systems enable direct measurement of canopy height, vertical stratification, and biomass. Li \cite{li_mapping_2021} demonstrated that integrating LiDAR with multispectral and hyperspectral data substantially improves multilayer mangrove structure mapping. Despite its accuracy, LiDAR remains spatially sparse and costly, limiting its operational use at national or global scales.

\subsection{Hyperspectral and Thermal Infrared Sensors}
Hyperspectral data support species-level discrimination and biochemical trait estimation, while thermal infrared data reveal physiological stress and hydrological anomalies \cite{farella_thermal_2022}, \cite{fu_cross-scenario_2025}. These modalities are underutilised in large-scale mangrove studies but offer significant potential for early stress detection and restoration monitoring.

\subsection{UAV-Based Observations}
UAV platforms provide ultra-high spatial resolution imagery for local-scale assessment and validation. While invaluable for monitoring restoration sites, UAV data lack scalability and require integration with satellite-based observations \cite{xu_review_2025}.

