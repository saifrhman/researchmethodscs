\section{Methodology}
The aim of this paper

\subsection{PRISMA Chart}
In the first stage of our scoping review, we identified records from five major 
databases. The initial search produced a combined set of studies, with the number 
of records obtained from each source presented in Figure X. Twenty-nine records 
were removed during initial organisation upon determining that Nature, being a 
publisher rather than a database, produced duplicated entries across our other 
information sources.

During screening, we identified a number of non-English 
records and further duplicates. Because our institution offers full-text access 
to all searched databases, none of the records were were excluded due to retrieval 
limitations.

The remaining studies were then divided among the research team, and each member 
screened titles and abstracts to assess their relevance to the research objectives. 
Studies were excluded if they did not involve geospatial data, if they relied solely 
on simple descriptive or statistical analyses, or if they did not use the modelling
approaches central to our review, such as Random Forest or CNN-based methods, 
they fell outside the scope of this study.