\section{Future Work: Toward Unified Multimodal Frameworks for Mangrove Restoration Prioritisation}

Building on the research gaps identified in this review, future work should prioritise a shift from isolated, region-specific modelling pipelines toward unified, multimodal, and generalisable computational frameworks for mangrove ecosystem assessment. Addressing these gaps requires coordinated advances in data integration, model design, evaluation methodology, and translational practice.

A primary direction for future work is the development of \textit{unified multimodal forest frameworks} that integrate complementary sensing modalities, including spatiotemporal optical imagery, SAR, LiDAR, thermal infrared, and multispectral or hyperspectral data. Because each modality captures distinct ecological attributes, multimodal integration enables holistic representations that cannot be achieved using any single data source. Multimodal integration enables the construction of holistic latent representations capable of supporting restoration prioritisation based on both current ecosystem condition and long-term resilience.

Recent advances in multimodal representation learning offer promising tools for achieving such integration. Attention-based architectures, including cross-attention multimodal transformers, allow heterogeneous data streams to interact at the feature or channel level, enabling dynamic weighting of modalities based on contextual relevance and data quality. This property is particularly valuable in mangrove environments, where sensor availability and informativeness vary spatially and temporally due to cloud cover, tidal state, and acquisition constraints. However, future research should remain architecture-agnostic, systematically comparing attention-based transformers, multimodal convolutional networks, graph-based models, and hybrid spatiotemporal pipelines under realistic data availability and domain shift conditions.

Addressing \textit{regional bias and limited generalisation} represents another critical research direction. Future frameworks should explicitly incorporate cross-region validation, transfer learning, and domain adaptation strategies to assess and improve model robustness across diverse ecological contexts. Pretraining multimodal models on globally distributed datasets, followed by region-specific fine-tuning using limited local data, represents a promising pathway for reducing dependence on data-rich regions and improving scalability. Such strategies are critical for ensuring that restoration prioritisation models remain robust under domain shift and can be deployed beyond data-rich regions.

Future work must also move beyond CNN-centric pipelines that struggle in fragmented mangrove landscapes. Exploration of multi-scale representations, object-centric models, weakly supervised learning, and semi-supervised approaches may improve detection of small or degraded mangrove patches while reducing reliance on extensive manual labelling. Such approaches are particularly important in urbanised or partially restored coastal environments where fragmentation is pronounced.

The challenge of observing submerged and underwater mangrove components further motivates integration of alternative sensing strategies. Combining satellite observations with UAV-based surveys, bathymetric data, and tide-aware data fusion can improve characterisation of intertidal and subtidal zones. Developing models that explicitly account for tidal state and water-column effects will be essential for accurate assessment of early-stage regeneration and below-canopy structures.

Finally, future research should prioritise \textit{restoration-oriented and participatory outputs}. Integrating field validation, in situ measurements, and local ecological knowledge into computational pipelines will be critical for ensuring that model outputs are actionable and context-aware. Restoration prioritisation frameworks should produce interpretable suitability indices, uncertainty estimates, and scenario-based projections that support decision-making by practitioners, policymakers, and local communities.