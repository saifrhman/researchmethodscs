\begin{abstract}

Mangrove ecosystems are some of the most productive on the planet, they prodive resources and are a key part of the fight against climate change. Surveying them accurately to gain insights that may guide conservation efforts and policy making is a paramount task. Ecological surveying has come a long way during the past decade. The advent of novel geographical scanning technologies like SAR, LiDAR and drones; equipped with various sensors that allow for infrared, light-spectrum, multispectral and hyperspectral imagery has come with the creation of massive Mangrove datasets. Advances deep learning algorithms has also opened the possibility to analyse these datasets under a new light. This scope review aims at evaluating the robustness of these new datasets and the ways they are analysed by the scientific community to further conservation efforts. Papers were retrieved using searching criteria that would narrow results to geographical data analysis that used models such as CNNs, ANNs, Random Forest algorithms, time-series analysis and otehr related techniques. It was found that across the evaluated publications there is strong regional bias among models, and a tranferable, multimodal approach is still lacking. Aslo the need of in-situ validation and cross referencing with local traditional ecological knowledge is stated as a completely lacking factor in the retrieved studies.

\end{abstract}
