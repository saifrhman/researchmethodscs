\begin{abstract}

Mangrove ecosystems provide critical ecosystem services, including carbon sequestration, coastal protection, and biodiversity support, yet they continue to experience widespread degradation. Advances in Earth observation and computational analysis have enabled increasingly detailed monitoring of mangrove extent, structure, and dynamics using multi-source geospatial data, alongside the growing adoption of machine learning methods. This scoping review synthesises the current landscape of geographical data and computational tools used to assess mangrove conservation status and restoration needs. We reviewed peer-reviewed studies applying geospatial data in combination with machine learning techniques and charted them by data sources, sensing modalities, analytical methods, evaluation practices, and application scope. While substantial progress has been made in data availability and modelling capability, the literature remains dominated by region-specific studies, with explicit cross-region evaluation or transfer learning rarely reported. Most analysis rely on individual sensing modalities, highlighting the absence of unified multimodal frameworks. Additional limitations include reduced robustness in fragmented landscapes, limited characterisation of submerged or intertidal zones, and minimal incorporation of in situ validation and local ecological knowledge. These gaps constrain the generalisability and restoration relevance of current approaches, underscoring the need for transferable, multimodal, and decision-oriented computational frameworks for mangrove ecosystem assessment and restoration prioritisation.

\end{abstract}
