\section{Current Data Availability and Dataset Robustness}
\subsection{Evolution of Global Mangrove Datasets}
Global mangrove datasets underpin large scale conservation assessments and policy initiatives. Early global products, such as the World Mangrove Atlas and the Landsat-based map by Giri et al. (2011), provided the first consistent estimates of global mangrove extent. While these datasets established critical baselines, they were static snapshots and did not capture temporal dynamics.

The introduction of time-series datasets marked a significant methodological shift. The Continuous Global Mangrove Forest Cover for the 21st Century (CGMFC-21) provided annual mangrove extent maps from 2000 to 2014 at 30 m resolution, enabling year-by-year analysis of loss and gain (Hamilton \& Casey, 2016). However, CGMFC-21 employed conservative mangrove definitions, resulting in substantially lower area estimates than other products and potentially under-representing sparse or degraded stands.

The Global Mangrove Watch (GMW) initiative represents the most comprehensive effort to date, integrating optical and SAR data to produce multi-epoch and near-global time-series products. The latest release, GMW v4.0, established a 10 m resolution global baseline for 2020 with reported overall accuracy exceeding 95\% (Bunting et al., 2022). This improvement significantly enhances the detection of narrow, fragmented, and small mangrove patches.

[Figure 1 placeholder: Timeline and comparison of major global mangrove datasets]

\subsection{Dataset Robustness and Uncertainty}
Despite methodological advances, global mangrove datasets remain subject to several limitations. Differences in sensor selection, classification algorithms, training data, and mangrove definitions lead to discrepancies in mapped extent, particularly near latitudinal range limits and in transitional ecosystems (Leal \& Spalding, 2024). Temporal gaps between dataset updates may obscure short-term disturbances or rapid regeneration events, while most global products prioritise extent over structural or functional attributes.

Furthermore, validation data are unevenly distributed geographically. Regions with limited field data often exhibit higher uncertainty, which can propagate into downstream analyses. These limitations highlight the need for complementary regional datasets, improved uncertainty quantification, and integration of structural and functional information for restoration planning.
