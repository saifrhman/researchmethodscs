\section{Current Data Availability and Dataset Robustness}
\subsection{Evolution of Global Mangrove Datasets}
Global mangrove datasets underpin large scale conservation assessments and policy initiatives. Early global products, such as the World Mangrove Atlas and the Landsat-based map \cite{giri_status_2011}, provided the first consistent estimates of global mangrove extent. While these datasets established critical baselines, they were static snapshots and did not capture temporal dynamics.

The introduction of time-series datasets marked a significant methodological shift. The Continuous Global Mangrove Forest Cover for the 21st Century (CGMFC-21) provided annual mangrove extent maps from 2000 to 2014 at 30 m resolution, enabling year-by-year analysis of loss and gain \cite{hamilton_creation_2016}. However, CGMFC-21 employed conservative mangrove definitions, resulting in substantially lower area estimates than other products and potentially under-representing sparse or degraded stands.

The Global Mangrove Watch (GMW) initiative represents the most comprehensive effort at the present time. It integrates optical and SAR data to produce multi-epoch and near-global time-series products. The latest release, GMW v4.0, established a 10 m resolution global baseline for 2020 with reported overall accuracy exceeding 95\% \cite{bunting_global_2022}. This improvement significantly enhances the detection of narrow, fragmented, and small mangrove patches. Table~\ref{tab:global_datasets} provides a comparative overview of major global mangrove datasets, summarising their release timelines, spatial resolution, temporal coverage, primary data sources, and the ecosystem attributes they capture. As per the latest report issued by the Global Mangrove Watch, this dataset retains its standing among the conservation community \cite{leal_state_2024}.

\begin{table*}[t]
\centering
\caption{Major global mangrove datasets and their key characteristics}
\label{tab:global_datasets}
\begin{tabular}{p{3.2cm} p{1.4cm} p{1.8cm} p{2.5cm} p{3.2cm} p{3.4cm}}
\toprule
Dataset & First Release & Spatial Resolution & Temporal Coverage & Primary Data Sources & Mangrove Attributes Captured \\
\midrule
Global Mangrove Forests Distribution & 2000 & $\sim$30 m & Circa 2000 & Landsat & Extent \\
CGMFC-21 \cite{hamilton_creation_2016} & 2014 & $\sim$30 m & 2000--2012 & Landsat-derived products & Annual extent and change \\
Global Mangrove Watch v2.0 \cite{noauthor_global_nodate}& 2019 & 20--25 m & 1996--2016 & ALOS PALSAR, Landsat & Extent and loss \\
Global Mangrove Watch v3.0 \cite{bunting_global_2022} & 2022 & $\sim$25 m & 1996--2020 & Multi-mission SAR and optical & Extent and temporal dynamics \\
Global Mangrove Canopy Height Maps \cite{simard_carbon_2024}& 2020+ & $\sim$30 m & Multi-year & DEM and optical integration & Canopy height and structure \\
GEDI \cite{duncanson_aboveground_2022} & 2019 & $\sim$25\,m LiDAR footprints & 2019--2023 & Spaceborne LiDAR (GEDI) & Canopy height, vertical structure, biomass proxies \\
Global Mangrove Soil Carbon Stocks \cite{maxwell_global_2023}& 2023 & $\sim$30 m & 2020 & Remote sensing and modelling & Soil organic carbon \\
\bottomrule
\end{tabular}
\vspace{0.5em}
\begin{minipage}{0.95\textwidth}
\footnotesize
\textit{Note: GEDI provides discrete LiDAR footprints rather than continuous spatial coverage and is therefore typically integrated with wall-to-wall mangrove extent datasets for analysis.}
\end{minipage}
\end{table*}

\subsection{Dataset Robustness and Uncertainty}

Despite methodological advances, global mangrove datasets remain subject to several limitations. Differences in sensor selection, classification algorithms, training data, and mangrove definitions lead to discrepancies in mapped extent. This is seen particularly near latitudinal range limits and in transitional ecosystems \cite{leal_state_2024}. Temporal gaps between dataset updates may obscure short-term disturbances or rapid regeneration events, while most global products prioritise extent over structural or functional attributes. This last point is an important observation, as it is ofter times that functional attributes that translate into ecological services (fishing grounds, water filtration, resources, etc.) are what truly moves communities and authorities to act pro or against conservation efforts. Assessing these attributes is thus of great relevance.

Lastly, validation data is unevenly distributed geographically. Regions with limited field data often exhibit higher uncertainty, which can propagate into downstream analyses. This can only be addressed with field work and completion of datasets. As stated by the Mangrove Global Watch, in locations where data fails, it is crucial to rely on local communities and traditional knowledge \cite{leal_state_2024}. This provides validation (or points out inaccuracies in the models), and makes the scientific efforts tangible to society. 
