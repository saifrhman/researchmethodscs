\section{Methodology}

\subsection{Article Selection Process}

In the first stage of this scoping review, records from four major databases were retrieved. The initial search produced a combined set of studies, with the number 
of records obtained from each source presented in Figure \ref{prisma}.

\begin{figure}
    \caption{As shown, records were retrieved from different databases, relevant studies were screened in Nature as well; though these were excluded as the same papers are expected to show up from the other databases.}
    \label{prisma}
    \includegraphics[width=7cm]{Figures/prisma.png}

\end{figure}

During screening, we identified a number of non-English 
records and further duplicates. Because our institution offers full-text access 
to all searched databases, none of the records were were excluded due to retrieval 
limitations.

The remaining studies were then divided among the research team, and each member screened titles and abstracts to assess their relevance to the research objectives. Studies were excluded if they did not involve geospatial data, if they relied solely on simple descriptive or statistical analyses, or if they did not use the modelling approaches central to our review, such as Random Forest or CNN-based methods, they fell outside the scope of this study.